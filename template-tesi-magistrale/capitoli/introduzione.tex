\chapter*{Introduzione}
\markboth{Introduzione}{}

Negli ultimi anni, i sistemi di raccomandazione hanno assunto un ruolo centrale nelle piattaforme digitali grazie all’evoluzione del Web e alla diffusione ubiqua dei servizi online. Essi rappresentano oggi uno strumento fondamentale per la gestione dell’overload informativo, generato dall’aumento esponenziale dei contenuti disponibili. Tali sistemi trovano applicazione in una vasta gamma di contesti, tra cui:

\begin{itemize}
    \item \textbf{Piattaforme di e-commerce}: per il suggerimento di prodotti simili o pertinenti, l’ottimizzazione delle pagine prodotto e del carrello, e l’aumento del tasso di conversione e del valore medio degli ordini.
    \item \textbf{Entertainment e piattaforme multimediali}: per raccomandare film, serie TV, musica e playlist, generare contenuti personalizzati e adattare dinamicamente le proposte in base al comportamento dell’utente.
    \item \textbf{Social network}: per l’ordinamento del feed in base alla pertinenza, il suggerimento di amici, pagine, gruppi o profili da seguire, e l’identificazione dei contenuti più rilevanti per ciascun utente.
    \item \textbf{News e informazione}: per la selezione degli articoli più rilevanti, la personalizzazione del feed di notizie in tempo reale e la raccomandazione di newsletter o approfondimenti tematici.
    \item \textbf{Digital advertising}: per la personalizzazione degli annunci, il targeting comportamentale e l’ottimizzazione del \textit{click-through rate}.
    \item \textbf{Formazione online ed e-learning}: per il suggerimento di corsi e lezioni pertinenti, l’adattamento del percorso formativo in base ai progressi e l’identificazione delle competenze da migliorare.
    \item \textbf{Servizi turistici e di prenotazione}: per il suggerimento di hotel, voli e destinazioni, il ranking personalizzato delle strutture e la gestione dei filtri basata su modelli comportamentali.
    \item \textbf{Finanza e banking}: per la raccomandazione di prodotti finanziari e assicurativi, l’analisi delle abitudini di spesa e i suggerimenti di investimento personalizzati.
    \item \textbf{Sanità digitale}: per il supporto decisionale ai medici, la raccomandazione di piani terapeutici basati sui dati del paziente e l’analisi predittiva dei rischi.
    \item \textbf{Retail fisico e smart shopping}: per suggerimenti personalizzati tramite app, l’ottimizzazione del layout dei negozi e la definizione di promozioni mirate.
\end{itemize}